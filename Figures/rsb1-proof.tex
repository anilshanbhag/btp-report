%And then the actual code for creating the figure:
\documentclass{standalone}
\usepackage{tikz}
\usetikzlibrary{arrows}
\begin{document}

% First, set the overall layout of the tree
% You might need to play with these sizes to ensure nothing overlaps.
\tikzstyle{level 1}=[level distance=1cm, sibling distance=2cm]
\tikzstyle{level 2}=[level distance=1cm, sibling distance=2cm]
\tikzstyle{level 3}=[level distance=1cm, sibling distance=2cm]
\tikzstyle{level 4}=[level distance=1cm, sibling distance=2cm]
\tikzstyle{level 5}=[level distance=1cm, sibling distance=2cm]
\tikzstyle{level 6}=[level distance=1cm, sibling distance=2cm]
\begin{tikzpicture}
\node {$\bowtie$}
    child{
        node{$C_{m}$}
        edge from parent
        node[left]{}
    }
    child[dashed]{
        node{$\bowtie$}
        child[solid]{
            node{$C_{2}$}
            edge from parent
            node[left]{}
        }
        child[solid]{
            node{$\bowtie$}
			child[solid]{
                node{$C_{1}$}
                edge from parent
                node[left]{}
            }
            child[solid]{
            		node{$\bowtie$}
				child{
                		node{$X$}
                		edge from parent
                		node[left]{}
            		}
            		child{
            				node{$R_{k}$}
                			edge from parent
                			node[right]{}
            		}
            }            
            edge from parent
            node[right]{}
        }
        edge from parent
        node[right]{}
    };
%Now I create the information set. Note that I utilize the names
%that I had previously assigned to nodes in my graph
%\draw [dashed](a)--(b);
%\draw [dashed](b)--(c);
\end{tikzpicture}
\end{document}